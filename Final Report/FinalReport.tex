\documentclass[10pt,twocolumn,letterpaper]{article}

\usepackage{iccv}
\usepackage{times}
\usepackage{epsfig}
\usepackage{graphicx}
\usepackage{amsmath}
\usepackage{amssymb}

\usepackage[pagebackref=true,breaklinks=true,letterpaper=true,colorlinks,bookmarks=false]{hyperref}

\iccvfinalcopy

\def\iccvPaperID{****} % *** Enter the ICCV Paper ID here
\def\httilde{\mbox{\tt\raisebox{-.5ex}{\symbol{126}}}}

% Pages are numbered in submission mode, and unnumbered in camera-ready
\ificcvfinal\pagestyle{empty}\fi
\begin{document}

%%%%%%%%% TITLE
\title{3D Photography - Final Report}

\author{Jan Rueegg\\
ETH Zurich\\
{\tt\small rggjan@gmail.com}
% For a paper whose authors are all at the same institution,
% omit the following lines up until the closing ``}''.
% Additional authors and addresses can be added with ``\and'',
% just like the second author.
% To save space, use either the email address or home page, not both
\and
Remo Meyer\\
ETH Zurich\\
{\tt\small meyerre@student.ethz.ch}
}

\maketitle
\thispagestyle{empty}

%%%%%%%%% ABSTRACT
\begin{abstract}
Some general explanation, maybe copy from proposal
\end{abstract}

\section{Classical ICP Algorithm}
\cite{fasticp}

The core of our implementation is the alignement of 2 pointclouds. 
We have devided this functionality into 4 subfunctions(Selection, Matching, Rejection and Minimization), described below. 
We didn't use any weighting of the points as this considered to not bring any advantage(fast icp paper reference) and would just add uneeded complexity.

\subsection{Selection}
First we have to choose which points we want to align. This process is called selection. 
We have choosen to sample the points randomly. 
This has several advantages as it is fast, easy to implement and each point has potentially an influence on the algortihm.

\subsection{Matching}
Initially we tried to match each selected point with it's nearest point in the target cloud. 
This generally worked, but had several downsides: First it was slow, especilly without an accelerated search structure (e.g. kd-tree). 
Furthermore it seems to be less stable than other matching approaches. 
So went for a projection based. We take the selected point translate it into the coordinatesystem of the other camera an backproject it with the second camera.
This is very fast as it is O(\# selected points) int opposite to O(\# selected points * points in cloud). 
We extended this approach with a search for compatible points, so we looked int the neighbourhood of the backprojected point for the point with the best matching color information(measuring euclidian distance in RGB space).

\subsection{Rejection}
To avoid too strong influence of outliers respectively wrong matches, we reject points that are to far away of each other. 
To decide which matches are too far away, we compute the average distance and then we discarded all matches that have a distance greater than this average times some constant factor(1.2).

\subsection{Minimization}
\label{minimization}
Now having all the matched points and their normals, we have to find a transformation matrix which minimizes the distance between those matched points.
As the camera is always the same an does not change(recalibrate) it may only have made a translation or a rotation, resulting in a matrix with 6 degrees of freedom.
Although the rotation part of the matrix is not really linear, but as we want to use SVD to solve this system we linearized the matrix

(Formula)

\cite{ptp}

When then solve this linear system with the eigen library(some ref to eigen library).
To compensate the error from linearizing the system, we first tried to run the minimization several times. 
This really brings some more accuracy for a single step. 
However as the solving of the system is one of the most time consuming tasks in this algorithm and most transformations have only a small part of rotation, it's better to make a new icp step than to make additional svd iterations.

\subsection{Optimizations}
We combined the Selection, matching and rejection step into one function. This allows us to have a constant number of samples, which then gives us more freedom in setting the number of samples.
As it can't occure anymore, that from time to time 80\% or more the samples are discarded. So testing showed, that fewer samples(around 50) reduce the computation time alot, while keeping the accuracy at a reasonable level.

\section{Additional Steps}
Besides implementing the steps described in the ''Fast ICP''\cite{fasticp} paper, we needed to implement some additional steps to get the algorithm working.

\subsection{Calibration}

\subsection{Normal estimation}

We decided to implement a point-to-plane error metric in the last step of the ICP Algorithm (section \ref{minimization}). To use that, we needed to define a plane respectively a normal at each of the matched points.

Since the kinect framework does not provide any normal information, we had to generate them ourselves. For that, we implemented and compared several different methods.

\subsubsection{Cross Product}
The first was to simply take the top/bottom/right/left neighbours and compute the normal with the crossproduct.
This approach has some limitations, as if only one of the 4 neighbours is a nan(due to the limitations in the kinect) the normal can't be compute. Furthermore it is very sensitive to noise, and only one outlier can make the normal unusable.

\subsubsection{Principal Components}
So we aimed at a more stable solution which is based on the computation of the principal coponents. 
The first 2 principal components describe the tangential direction of the plane while the 3th is the normal component.
Compared to the previous solution this is much slower, but much more stable and an arbitrary number of neighbouring points may be chosen.
To be able to decide, how to make the trade-off between the number of computations and the accuracy, we made some visualizations of the computed normals.
We varied the method and the size of the considered neighbourhood respectively the offset between the samples.

\subsubsection{Extended Cross Product}

(some images of the described tests)

Finally we aimed for the .... method with the parameters ... , because ....

\subsection{Usage of Color and Normals}

\section{Results}

\section{Discussion}

\section{Used libraries}
\begin{description}
\item[Eigen 2]
A lightweight C++ template library for vector and matrix math,
a.k.a. linear algebra.

- - Point and matrix type declarations
- - svd solver for normal computation and minimization

\href{http://eigen.tuxfamily.org}{http://eigen.tuxfamily.org}

\item[Robot Operating System (ROS)]
A meta-operating system for robots. It provides
language-independent and network-transparent communication for a
distributed robot control system.

- - register to kinect data

\href{http://ros.org}{http//ros.org}

\item[Point Cloud Library (PCL)]
A comprehensive open source library for n-D Point Clouds and 3D geometry processing.

- - point cloud type definitions

\href{http://pointclouds.org}{http://pointclouds.org}

\item[The ROS OpenNI project]
ROS OpenNI is an open source project focused on the integration of the PrimeSense sensors with ROS.

-- kinect

\href{http://www.ros.org/wiki/ni}{http://www.ros.org/wiki/ni}

\end{description}

{\small
\bibliographystyle{ieee}
\bibliography{egbib}
}

\begin{thebibliography}{99}
\bibitem{fasticp}
Szymon Rusinkiewicz and Marc Levoy.
\emph{Efficient Variants of the ICP Algorithm}.
Proceedings of the International Conference on 3-D Digital Imaging and
Modeling (3DIM), pp. 145–152, 2001.

\bibitem{ptp}
Kok-Lim Low.
\emph{Linear Least-Squares Optimization for
Point-to-Plane ICP Surface Registration}.
Technical Report TR04-004, Department of Computer Science, University of North Carolina at Chapel Hill, February 2004.

\end{thebibliography}


\end{document}
